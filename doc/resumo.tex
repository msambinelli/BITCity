{
\Large
\begin{center}
\textbf{Resumo}
\end{center}
}

Concorrência é quando duas ou mais unidades são executadas em
simultâneo. Como estas unidades podem compartilhar os mesmo recursos, isto
pode acarretar em alguns problemas: condição de corrida e deadlock. Assim,
para evitar estes problemas, se torna necessário encontrar uma forma
de sincronizar as unidades, consequentemente mantendo consistência nas
informações utilizadas.  Erros são comuns em tempo de execução e as
aplicações devem estar preparadas para lidar com eles.  Com o uso de
exceções, podemos interromper o fluxo normal do programa quando ocorrer
uma exceção e, assim, podemos trata-la.

\quad\\
\quad\\
\textit{Palavras-chave}: Concorrência, Exceções, Java

\pagebreak
