{
\Large
\begin{center}
\textbf{Resumo}
\end{center}
}

Concorrência ocorre quando duas ou mais unidades são executadas em
simultâneo. Como estas unidades podem compartilhar os mesmo recursos, isto
pode acarretar em problemas como condição de corrida e deadlock.
Para evitar estes problemas, torna-se necessário encontrar uma forma
de sincronizar as unidades, mantendo a consistência nas
informações utilizadas. Erros são comuns em tempo de execução e as
aplicações devem estar preparadas para lidar com eles.  Com o uso de
exceções, pode-se interromper o fluxo normal do programa quando ocorrer
uma exceção e tratá-la.

Para estudar os problemas e características da programação concorrente,
a linguagem de programação Java foi utilizada no desenvolvimento de um
aplicativo chamado de \textbf{Bitcity}. Essa aplicação trabalha com um
modelo textual, construindo um mundo a partir da mesma e controlando
acesso a recursos compartilhados.

\quad\\
\quad\\
\textit{Palavras-chave}: Concorrência, Exceções, Java

\pagebreak
