\chapter{Concorrência e Exceções}
\label{cha:concexec}

Há alguns níveis de concorrência: concorrência de instrução de máquina, onde são
executadas mais do que uma instrução de máquina por vez; concorrência de
instrução, onde são executadas mais do que uma instrução por vez;
concorrência em nível de unidade, onde são executadas mais do que uma
unidade de subprograma por vez; nível de programa, onde são executados mais
do que um subprograma por vez. 

Uma unidade pode ser executada concorrentemente de duas formas: fisicamente
ou logicamente. Na concorrência física duas ou mais unidades do mesmo
programa são executadas simultaneamente, sendo necessário a existência de
mais de um processador para que isso seja possível. Quando não há a
disponibilidade de múltiplos processadores pode ocorrer a concorrência
lógica, onde o programador e o programa supõem a existência de múltiplos
processadores fornecendo a concorrência física, enquanto na verdade a
execução das unidades está sendo realizada de forma intercalada. Esta
ilusão é semelhante à ilusão de execução simultânea oferecida a diferentes
usuários de um sistema de computadores com multiprogramação.

Para visualizar um fluxo de execução de um programa, podemos imaginar uma
linha ({\it thread}) disposta sobre o código fonte. Cada linha representa um
fluxo de execução dentro do programa. Programas executando em computadores
multiprocessadores podem executar cada fluxo (linha) em um processador
diferente simultaneamente. Na concorrência lógica,  múltiplas linhas do
programa são mapeadas em apenas uma, fazendo com que o programa tenha
virtualmente múltiplas linhas. 

Uma tarefa é uma unidade de um programa que pode estar em execução
concorrente com outras unidades do mesmo programa. Cada tarefa de um
programa pode oferecer uma linha de controle \cite{sebesta}. Diferente de
um subprograma, as tarefas podem ser implicitamente inicializadas e, além
disso, quando uma unidade invoca uma tarefa, ela não precisa esperar que
está termine para continuar a sua execução.

Como todas as linhas compartilham os mesmos dados é necessário a utilização
de algum mecanismo de sincronização para manter a consistência desses
dados. Atualmente o método de se fazer sincronização mais utilizada é
a exclusão mútua \cite{ARTOFMPP, MCRT}. 

Este Capítulo está organizado da seguinte maneira: Na Seção \ref{sec:prob}
será abordado alguns problema comuns na programação paralela. A Seção
\ref{sec:mecanismo} abordará alguns mecanismos utilizados para se fazer
a sincronização. Por último, a Seção \ref{sec:except} apresenta 
conceitos a cerca das exceções.

\section{Problemas Comuns na Computação Paralela}
\label{sec:prob}

\subsection{Condição de Corrida}

Quando duas tarefas trabalham juntas, eles podem compartilhar algum recurso
em comum. Uma condição de corrida ocorre quando duas ou mais tarefas
necessitam efetuar operações em um conjunto de dados compartilhado e o 
resultado destas operações dependem da ordem em que essas operações
são executadas \cite{tanen}.

\subsection{Deadlock}

Um conjunto de tarefas estará em situação de \textit{deadlock} se todas as
tarefas pertencentes ao conjunto estiverem esperando por um evento que somente
tarefa desse mesmo conjunto possa realizar. Como todas as tarefas estarão
esperando, nenhuma delas desencadeará qualquer um dos eventos que a outra
esta esperando e, assim, todas as atividades continuam a esperar para sempre.

Para que ocorra um \textit{Deadlock}, deve haver quatro condições satisfeitas:

\begin{itemize}
\item \textit{Condição de exclusão mútua}: Em um determinado instante,
cada recurso está em uma de duas situações: ou associado a uma única tarefa
ou disponível. 
\item \textit{Condição de posse e espera}: Tarefas que,
em um determinado instante, retêm recursos concedidos anteriormente podem
requisitar novos recursos. 
\item \textit{Condição de não preempção}:
Recursos concedidos previamente a uma tarefa não podem ser forçadamente
tomados - eles deve ser explicitamente liberados pela tarefa que os retém.
\item \textit{Condição de espera circular}: Deve existir um encadeamento
circular de duas ou mais tarefas; cada uma delas encontra-se à espera de
um recurso que está sendo usado pelo membro seguinte dessa cadeia.
\end{itemize}

Todas essas quatro condições deve estar presentes para que um
\textit{deadlock} ocorra. Se faltar uma delas, não ocorrerá
\textit{deadlock}.

Em geral, quatro estratégias são usadas para tratar \textit{deadlocks}:

\begin{itemize}
\item Ignorar por completo o problema.
\item Detecção e recuperação. Deixar os \textit{deadlocks} ocorrer,
detectá-los e agir.
\item Anulação dinâmica por meio de uma alocação cuidadosa de recursos.
\item Prevenção, negando estruturalmente algumas das condições necessárias
para gerar um \textit{deadlock}.
\end{itemize}

\subsection{Inversão de Prioridade}

Inversão de prioridade descreve um caso geral onde uma tarefa de baixa
prioridade guarda um recurso compartilhada para uso exclusivo que é
requerido por uma tarefa de alta prioridade \cite{intel}. A tarefa de alta
prioridade é então forçada a ficar bloqueada enquanto a tarefa de baixa
prioridade não liberar o recurso.

%\subsubsection{Livelock}
%
% WTF?
%Um \textit{Livelock} é semelhante a um \textit{Deadlock}, com exceção dos estados dos processos envolvidos na 
%\textit{Livelock} mudam constantemente em relação um ao outro, nenhum progresso \textit{Livelock} é um caso 
%especial de esgotamento de recursos.
%
%Um exemplo do mundo real de \textit{Livelock} ocorre quando duas pessoas se encontram em um corredor 
%estreito, e cada um tenta ser educada, movendo de lado para  dar passagem a outra pessoa, 
%mas eles acabam movendo de um lado para o outro sem fazer nenhum progresso porque ambos 
%repetidamente movem na mesma direção e ao mesmo tempo.
%

\section{Mecanismos de Sincronização}
\label{sec:mecanismo}

\subsection{Lock}

\textit{Lock} é uma solução de software para tentar impedir que duas
tarefas acessem os mesmos dados compartilhados simultaneamente,
implementado assim a exclusão mutua \cite{MCRT}. Considere que haja uma variável
compartilhada (\textit{lock}), inicialmente contendo o valor 0. Quando uma
tarefa deseja utilizar um recurso compartilhado, ela primeiro testa 
o valor da variável \textit{lock}. Se \textit{lock} for 0, a tarefa
altera essa variável para 1 e utiliza o dado compartilhado. Se \textit{lock}
já estiver com o valor 1, a tarefa simplesmente aguardará até que ela
se torne 0. Assim, um 0 significa que nenhuma tarefa está utilizando o
recurso e um 1 indica que existe uma tarefa que está utilizando o recurso.

\subsection{Lock reentrante}

\textit{Lock} reentrante é um \textit{lock} que pode ser readquirido pela
\textit{thread} que o possui sem causar \textit{deadlock} com ela mesma.

\subsection{Spin}

\textit{Spin locks} são um tipo especial de \textit{lock} projetados
para trabalhar em um ambiente multiprocessador. Se uma tarefa encontra
o \textit{spin lock} aberto, ele adquiri o \textit{lock} e acessa os
dados compartilhados.  Caso contrário ela fica presa em um {\it loop}.
A vantagem desse método é que a maioria dos recursos ficam trancadas por
milissegundos apenas, seria mais custoso liberar a \textit{CPU} e tenta-la
readquiri-la posteriormente.

%\subsection{Block}
%….


\subsection{Semáforos}

Em 1965 Dijkstra sugeriu usar uma variável inteira para contar o número de
sinais para despertar salvos para uso futuro.  De acordo com a proposta dele,
foi introduzido um novo tipo de variável chamado semáforo. Um semáforo
poderia conter o valor 0 – indicando que nenhum sinal de acordar foi
salvo – ou algum valor positivo se um ou mais sinais de acordar estivessem
pendentes \cite{TANE}.

Dijkstra propôs a existência de duas operações: \texttt{down} e
\texttt{up}. A operação \texttt{down} sobre um semáforo verifica se seu
valor é maior que 0, caso seja, o decrementa em uma unidade e prossegue
com a sua execução.  Se o valor do semáforo for 0, a tarefa será posta
para dormir, sem terminar a operação\texttt{down}. Garante-se que, uma vez
iniciada uma operação de semáforo, nenhuma outra tarefa pode ter acesso
ao semáforo até que a operação tenha terminado ou sido bloqueada. Essa
atomicidade é absolutamente essencial para resolver os problemas de
sincronização e evitar condições de corrida.

A operação \texttt{up} incrementa o valor de um dado semáforo. Se uma
ou mais tarefas estivessem dormindo naquele semáforo, incapacitadas de
terminar sua operação \texttt{down}, uma delas será escolhida pelo sistema
(por exemplo, aleatoriamente) e será dada a permissão para terminar sua
operação \texttt{down}. Portanto, depois de um \texttt{up} em um semáforo
com tarefas dormindo nele, o semáforo permanecerá 0, mas haverá uma tarefa
a menor dormindo nele.

\subsection{Monitor}

Para facilitar a escrita correta de programas, Hoare e Brinch Hansen propuseram
uma unidade básica de sincronização de alto nível chamada monitor.
Um monitor é uma coleção de procedimentos, variáveis e estruturas de dados,
tudo isso agrupado em um tipo especial de módulo. As tarefas podem invocar
os procedimentos de um monitor quando quiserem, mas não podem ter acesso
direto às estruturas internas de dados ao monitor a partir de procedimentos
declarados fora do monitor.

Os monitores apresentam uma propriedade importante que os torna úteis
para realizar a exclusão mútua: somente uma tarefa pode estar ativa em um
monitor em um dado momento. O monitor é uma construção da linguagem de
programação. Em geral, quando uma tarefa chama um procedimento do monitor,
algumas das primeira instruções do procedimento verificarão se qualquer
outra tarefa está atualmente ativa dentro do monitor. Se estiver, a tarefa
que o invocou será suspensa até que a outra tarefa deixe o monitor. Se nenhuma
outra tarefa estiver usando o monitor, a tarefa que o invocou poderá entrar.


\subsection{Memória Transacional}


Memória Transacional tem sido muito discutida atualmente, e o seu grande
interesse é pelo fato desta eliminar os problemas comumente associados com
locks, tais como:  \cite{TMASFKDS}{\it convoying}, \cite{TANE} {\it deadlock} e
\cite{intel} {\it inversão de
prioridade}. Um sistema de memória transacional utiliza \cite{bernstein} transações como
um mecanismo para abstrair a computação paralela. Uma transação é qualquer
operação que garante as propriedades ACID:

\begin{itemize}
    \item Atomicidade
    \item Consistência
    \item Isolamento
    \item Durabilidade
\end{itemize}

Sendo apenas as propriedades ACI utilizadas pela Memória transacional. Como
um exemplo de sistema de memória transacional podemos citar  \cite{LIGHTTRANS},
que estendeu a linguagem \textit{Java} para suportar transações e
\cite{TOTAPI}, que implementou uma API baseada na famosa biblioteca OpenMP.

\section{Exceções}
\label{sec:except}

Uma exceção é um evento, normalmente associado a um erro, e que ao ser
disparado interrompe o fluxo normal de execução da aplicação. Exceções
ocorrem durante a execução de aplicativos, como por exemplo divisão
por zero ou acesso a posições inválidas em \textit{arrays}.

Erros são comuns em tempo de execução e as aplicações devem estar
preparadas para tratá-los sem abortar o programa e perder os dados dos
usuários. O conceito de tratamento de exceções permite ao programador
``lidar'' com o problema (tratando-o) e permitindo ao programa continuar
sua execução.

As vantagens no uso de exceções são: 

\begin{itemize}
\item Separação do código regular (fluxo normal) do código de tratamento de exceções
\item Agrupamento e diferenciação de erros (e seus respectivos tratamentos)
\item Obrigatoriedade de tratamento (exceções não podem ser ignoradas)
\end{itemize}


%\subsection{Seções Críticas}
%
%Durante uma parte do tempo, um processo está ocupado fazendo computações
%internas e outras coisas que não acarretam em condições de disputa. Contudo,
%algumas vezes, um processo precisa ter acesso à memória compartilhada ou a
%recursos compartilhados ou tem que fazer outras coisas críticas que podem
%ocasionar disputas. Aquela parte do programa em que há acesso à memória
%compartilhada é chamado de seção crítica.
