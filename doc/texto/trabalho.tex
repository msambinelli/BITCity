\chapter{Bitcity}

XXX Aqui tem um parágrafo falando que usou Java porque é uma linguagem
bem legal que tem um modelo de concorrência e tratamento de exceções
joia e que para explorar tais recursos foi feita uma aplicação,
chamada de \textbf{Bitcity}.XXX

A \textbf{Bitcity} parte de um modelo textual para projetar um mundo
onde retângulos azuis representam carros que respeitam os semáforos
mas que não apreciam congestionamento, esporadicamente
retângulos brancos surgem e representam uma ambulância que age como um
coletor de recursos, folhas crescem em árvores e a chuva faz as mesmas
caírem.

O Modelo \ref{short-one} apresenta todos os elementos que podem ser
encontrados no modelo textual.

\begin{model}
\begin{verbatim}
9 18 3 2 15
######           #
*>  -#           #
$..$-#           # 
$..$A#############
*> A-     B>>
###$ ######B #####
  #$ #    #  #    
  #$ #    #+ #####
###$ #    #+    <*
\end{verbatim}
  \caption{Exemplo demonstrativo \label{short-one}}
\end{model}

A primeira linha contém quatro inteiros que descrevem, respectivamente,
quantidade de linhas, quantidade de colunas, número de pontos de
partida, número de conjuntos de semáforos a serem sincronizados e a
quantidade máxima de carros no mundo num dado instante.

O caractere \verb!`#'! representa calçada, \verb!`$'! representa
vegetação rasteira, \verb!`.'! uma árvore que ganha e perde folhas ao
longo da execução, uma letra qualquer que se repete indica semáforos
que pertencem a um mesmo conjunto. Os pontos de partida, que ficam
responsáveis por indicar os locais onde carros surgem no mundo, são
representados por \verb!`*'!. Os símbolos \verb!`+'!, \verb!`-'!,
\verb!`>'! e \verb!`<'! indicam, respectivamente, para cima, para
baixo, esquerda e direita e são utilizados pelos veículos para
movimentação no mundo.

O modelo é projetado de forma a simplificar a construção do mundo e
também o \textit{parsing} do mesmo. Todo ponto de partida precisa,
necessariamente, que uma direção inicial esteja presente ou
imediatamente a sua direita, ou a esquerda, ou em cima ou
embaixo. Essa direção define a direção de deslocamento inicial de
todos os carros que partirem daquele ponto. Os carros possuem uma
unidade de visão à sua frente, considerando sua direção atual. Esta
visão é utilizada para tomada de seis decisões:
\begin{description}
\item[Mudar direção atual]: O elemento à frente contém um dos
  caracteres que indicam uma direção e à frente deste outro caractere,
  considerando a direção que representa, o símbolo se repete. Esse
  caso ocorre no Modelo \ref{short-one} na linha 2, coluna 5.
\item[Escolher entre mudar de direção ou não]: O elemento à frente
  contém um dos caracteres que indicam uma direção e à frente deste
  outro caractere, considerando a direção que representa, o símbolo
  não se repete. Situação da linha 5, coluna 5 no Modelo \ref{short-one}.
\item[Parar no semáforo]: Veículo, exceto ambulância, depara-se com
  sinal vermelho e obrigatoriamente para.
\item[Buzinar]: O sinal está aberto, porém existe um veículo a sua
  frente que está parado.
\item[Sair do mundo]: O veículo atingiu o limite do mundo. No Modelo
  \ref{short-one} essa situação pode ocorrer somente ao atingir uma
  linha após a 9 na coluna 5 e também uma coluna após a 18 na linha 5.
\item[Deslocar-se]: Posição à frente não contém nenhum elemento
  especial, o veículo pode continuar deslocando-se seguindo sua
  direção corrente.
\end{description}

A aplicação espera iniciar sua execução com um modelo que se adeque as
especificações a cima. Caso isso não ocorra, a classe \verb!Parser!
fica responsável por detectar o problema e levantar uma exceção. A
classe \verb!Application! fica responsável por tratar esta exceção,
sendo exibido o erro encontrado e abortando a execução da aplicação.
Recebendo um modelo correto, a aplicação trata de renderizar o mundo
especificado. ...
