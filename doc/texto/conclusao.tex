\chapter{Conclusão}
\label{cha:conclusao}

Com os processadores atuais, que possuem diversos núcleos, com a própria
modelagem de alguns problema e com a profecia de que a computação paralela
será a próxima etapa na evolução dos computadores \cite{TM}, é uma grande
desvantagem de uma linguagem que se diga moderna não possuir nenhum
mecanismo para abstrair a computação paralela, mesmo ao custo do projeto da
linguagem se tornar mais complexo. A aplicação do Capítulo
\ref{cha:bitcity} se implementada sequencialmente seria muito mais
complexa e poderia ter um desempenho muito pobre devido ao grande esforço
computacional exigido para atualizar a tela da aplicação.

No que diz respeito à exceções, a desvantagem de ter suporte a
exceções na linguagem, também é o fato do projeto da linguagem e o seu projeto
ficarem mais complexos e da necessidade de implementar um manipulador
de exceções. Por outro lado, suporte a exceções pela linguagem torna
mais fácil a manipulação de erros, pois evita a necessidade do uso de
uma variável \texttt{status} e de uso de um condicional sempre que se
deseja verificar a ocorrência de um evento.

Uma ilustração de como o suporte de linguagem à exceções pode simplificar
o código pode ser visto em como um código C faria o tratamento da entrada de
arquivos. Digamos que um programa C precise abrir 10 arquivos diferentes, pelo
fato de C não suportar exceções é necessário checar uma variável de {\it
status}, o retorno da função, para saber se a operação foi realizada com
sucesso, o que obrigaria o desenvolvedor a colocar 10 testes condicionais para
checar se cada uma das operações de abrir arquivo foi realizada com
sucesso. Se a linguagem de programação C suporta-se exceções como a
linguagem Java por exemplo, bastaria apenas colocar essas 10 operações
dentro de uma cláusula \texttt{try} e declarar uma cláusula \texttt{catch}
para manipular exceções as exceções geradas.
