\chapter{Concorrência em Java}
\label{cha:concjava}

\section{Interface Runnable e classe Thread}

A linguagem de programação \texttt{Java} fornece diretamente suporte a
aplicações \textit{multithread}, permitindo a criação de linhas de
execução por meio de extensão da classe \verb!Thread! ou implementação
da interface \verb!Runnable!.

Não há diferenças, no que diz respeito a concorrência, entre um método
e outro. Porém, como a linguagem disponibiliza suporte apenas a
herança simples, extender a classe \verb!Thread! significa que não
será possível herdar funcionalidades de qualquer outra classe. Para
resolver esse problema, a aplicação pode criar uma hierarquia de
classes ou, então, implementar a interface \verb!Runnable!.

Indiferente do método utilizado, o corpo da classe deve conter a
implementação de um método \verb!run!. Este será invocado ao iniciar a
\textit{thread} por meio do método \verb!start!. O método \verb!run!
serve um propósito similar ao da função \verb!main! na linguagem
\texttt{C}, ou mesmo \texttt{Java}, ou seja, ao encerrar a execução do
mesmo a linha de execução é considerada como encerrada e não pode mais
ser reiniciada.

\section{Synchronized}

Em JAVA a exclusão mutua é implementada utilizando-se o modificador
\texttt{synchronized}. \texttt{synchronized} faz com que os métodos sejam
executados inteiramente um atrás do outro sobre o mesmo objeto, ou seja,
quando um método iniciar a sua operação sobre um objeto ele irá terminar
antes que qualquer outro método inicie suas operações sobre o mesmo objeto.

Um objeto que possui métodos sincronizados deve possuir uma fila para
guardar os métodos que tentaram acessar o objeto enquanto este estava sendo
utilizado por um outro método sincronizado. Assim quando o método que
atualmente está utilizando o objeto terminar as suas operações sobre esse
, deve-se permitir que um próximo método da fila de espera realize suas
operações sobre o objeto. 

Abaixo é mostrado um pequeno exemplo retirado de \cite{sebesta}. O exemplo
mostra dois métodos modificados com \texttt{synchronized} e desta forma
impedindo que um interfira na computação do outro.

\begin{lstlisting}
class GerenciaREtentor {
    private int[100] retentor;
    ...
    public synchronized void deposita(int item) { ... }
    public synchronized int busca() { ... }
    ...
}
\end{lstlisting}

O modificador \texttt{synchronized} também permite sincronizar apenas um
segmento de código, isso é útil quando o trecho de código que trabalha com
dados compartilhados é muito pequeno em relação ao tamanho do método. Isso
pode ser feito com a denominada instrução sincronizada:

\begin{lstlisting}
synchronized(expression){
    instruction
}
\end{lstlisting}

Uma expressão que é avaliada para um objeto e quando
satisfeita o objeto é bloqueado durante a execução do corpo de
\texttt{synchronized}.

\section{Exception}

A linguagem de programação Java utiliza exceções para tratar erros e
demais eventos excepcionais. Uma exceção é um evento que ocorre durante
a execução de um programa que interrompe o fluxo normal das instruções
\cite{except}.

O primeiro passo na construção de um manipulador de exceção é colocar
o código que pode lançar uma exceção dentro de um bloco \texttt{try}. Em geral,
um bloco \texttt{try} é semelhante ao seguinte.

\begin{lstlisting}

try {
    code
}
catch and finally blocks . . .
    
\end{lstlisting}


Se ocorrer uma exceção no bloco \texttt{try}, a exceção é tratada por um
manipulador de exceção a ele associado. Para associar um manipulador de
exceção com um bloco \texttt{try}, você definir colocar um bloco
\texttt{catch}.

Você associa manipuladores de exceção, com um bloco \texttt{try}, fornecendo
um ou mais blocos \texttt{catch} logo após o bloco \texttt{try}. Nenhum
código pode ser entre o final do bloco \texttt{try} e no início do primeiro bloco % WTF?!
\texttt{catch}.

\begin{lstlisting}

try {
     
} catch (ExceptionType name) {
     
} catch (ExceptionType name) {
     
}  

\end{lstlisting}
