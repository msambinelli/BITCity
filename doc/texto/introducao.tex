\chapter{Introdução}

Alguns problemas possuem uma solução concorrente mais natural do que a
sequencial, é o caso de muitos problemas de simulação, onde diversas
variáveis do problema são modificadas independentemente e de forma
concorrente. É o caso da simulação de uma cidade, um caro $i$ não se move
para depois um carro $j$ se mover, eles se movem de forma independente um
do outro.

Além de ser a solução mais natural para alguns tipos de problema, muitos
computadores atualmente possuem múltiplos processadores, e dessa forma, 
uma aplicação pode aumentar sua performasse usando a concorrência em tais
arquiteturas.

% xxx falar sobre excessoes


Este trabalho está dividido da seguinte forma. No Capítulo
\ref{cha:concexec} é feita uma pequena introdução a concorrência, seus
principais problema e seus mecanismos de sincronização, e exceções. O
Capítulo \ref{cha:concjava} aborda os principais mecanismos oferecidos pela
linguagem JAVA para se implementar a concorrência e manipular exceções. O
Capítulo \ref{cha:bitcity} apresenta um simulador de uma cidade que faz uma
amplo uso dos mecanismo de concorrência, sincronização e tratamento de
exceções em JAVA. No Capitulo \ref{cha:conclusao} é apresentada as
vantagens e desvantagens do suporte pela linguagem a concorrência e
exceções. 
