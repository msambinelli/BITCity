\chapter{Introdução}

Alguns problemas possuem uma solução concorrente mais natural do que a
sequencial, é o caso de muitos problemas de simulação, onde diversas
variáveis do problema são modificadas independentemente e de forma
concorrente. É o caso da simulação de uma cidade, um carro $i$ não se move
para depois um carro $j$ se mover, eles se movem de forma independente um
do outro.

Além de ser a solução mais natural para alguns tipos de problema, muitos
computadores atualmente possuem múltiplos processadores e, dessa forma, 
uma aplicação pode aumentar sua performance usando a concorrência em tais
arquiteturas.

Outro recurso importante que muitas linguagens modernas implementam é o
tratamento de exceções. Uma exceção é qualquer evento errôneo, ou não,
que seja detectável por \textit{hardware} ou \textit{software} e que possa
exigir processamento especial \cite{sebesta}. O suporte de exceções pela
linguagem aumenta a confiabilidade do programa e força o programador
a desenvolver pensando nos casos extremos.

Este trabalho está dividido da forma seguinte. No Capítulo
\ref{cha:concexec} é feita uma pequena introdução a concorrência, seus
principais problemas, seus mecanismos de sincronização e exceções. O
Capítulo \ref{cha:concjava} aborda os principais mecanismos oferecidos pela
linguagem Java para implementar a concorrência e manipular exceções. O
Capítulo \ref{cha:bitcity} apresenta um simulador de uma cidade que faz
uso dos mecanismo de concorrência, sincronização e tratamento de
exceções em Java. No Capítulo \ref{cha:conclusao} é apresentada as
vantagens e desvantagens do suporte pela linguagem a concorrência e
exceções. 
