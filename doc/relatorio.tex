\documentclass[brazil,12p,A4]{abnt}
\usepackage{paralist}
\usepackage{graphicx,url,subfigure}
\usepackage[brazil]{babel}
\usepackage[utf8]{inputenc}
\usepackage[T1]{fontenc}
\usepackage{ae}
\usepackage[alf,bibjustif,abnt-etal-list=0]{abntcite}
\usepackage{alltt}
\usepackage{booktabs}
\usepackage{float}
\usepackage{amssymb}
\usepackage{amsmath}
\usepackage{subfigure}
\usepackage{fancyvrb}
\usepackage{rotating}
\usepackage{multirow}

\usepackage{color}
\usepackage{caption}
\usepackage{listings}
\lstset{
  basicstyle=\footnotesize\ttfamily,
  stepnumber=1,
  numbersep=5pt,
  tabsize=2,
  extendedchars=true,
  breaklines=true,
  keywordstyle=\bfseries,
  showspaces=false,
  showtabs=false,
  xleftmargin=17pt,
  framexleftmargin=17pt,
  framexrightmargin=5pt,
  framexbottommargin=4pt,
  showstringspaces=false,
}
\lstloadlanguages{
  C, Java
}

\DeclareCaptionFont{black}{\color{black}}
\DeclareCaptionFormat{listing}{#1#2#3}
\captionsetup[lstlisting]{format=listing, textfont=black,
  }

\newfloat{model}{thp}{lop}
%\captionof{model}{Modelo}
\floatname{model}{Modelo}

\newcommand{\up}[1]{\raisebox{1.5ex}[0pt]{#1}}
\newcommand{\doctitulo}{Bitcity}
\newcommand{\docautor}{Guilherme Enoc Egas\\Guilherme Henrique Polo Gonçalves\\
	Maycon Sambinelli}


\begin{document}

\baselineskip=.7cm


\vspace{4cm}

\begin{center}

\textbf{UNIVERSIDADE ESTADUAL DE MARINGÁ}

\textbf{CENTRO DE TECNOLOGIA}

\textbf{DEPARTAMENTO DE INFORMÁTICA}

\vspace{4cm}

\textbf{\MakeUppercase{\doctitulo}}

\vspace{1cm}

%\MakeUppercase{\docautor}
\docautor

\vspace{1cm}

\end{center}

\vspace{10cm}

\begin{center}
\centering
\textbf{Maringá - Paraná}

\textbf{2010}
\end{center}


\pagebreak

{
\Large
\begin{center}
\textbf{Resumo}
\end{center}
}

Concorrência ocorre quando duas ou mais unidades são executadas em
simultâneo. Como estas unidades podem compartilhar os mesmo recursos, isto
pode acarretar em problemas como condição de corrida e deadlock.
Para evitar estes problemas, torna-se necessário encontrar uma forma
de sincronizar as unidades, mantendo a consistência nas
informações utilizadas. Erros são comuns em tempo de execução e as
aplicações devem estar preparadas para lidar com eles.  Com o uso de
exceções, pode-se interromper o fluxo normal do programa quando ocorrer
uma exceção e tratá-la.

Para estudar os problemas e características da programação concorrente,
a linguagem de programação Java foi utilizada no desenvolvimento de um
aplicativo chamado de \textbf{Bitcity}. Essa aplicação trabalha com um
modelo textual, construindo um mundo a partir da mesma e controlando
acesso a recursos compartilhados.

\quad\\
\quad\\
\textit{Palavras-chave}: Concorrência, Exceções, Java

\pagebreak


\tableofcontents


\pagebreak

\chapter{Introdução}

Alguns problemas possuem uma solução concorrente mais natural do que a
sequencial, é o caso de muitos problemas de simulação, onde diversas
variáveis do problema são modificadas independentemente e de forma
concorrente. É o caso da simulação de uma cidade, um caro $i$ não se move
para depois um carro $j$ se mover, eles se movem de forma independente um
do outro.

Além de ser a solução mais natural para alguns tipos de problema, muitos
computadores atualmente possuem múltiplos processadores, e dessa forma, 
uma aplicação pode aumentar sua performasse usando a concorrência em tais
arquiteturas.

% xxx falar sobre excessoes


Este trabalho está dividido da seguinte forma. No Capítulo
\ref{cha:concexec} é feita uma pequena introdução a concorrência, seus
principais problema e seus mecanismos de sincronização, e exceções. O
Capítulo \ref{cha:concjava} aborda os principais mecanismos oferecidos pela
linguagem JAVA para se implementar a concorrência e manipular exceções. O
Capítulo \ref{cha:bitcity} apresenta um simulador de uma cidade que faz uma
amplo uso dos mecanismo de concorrência, sincronização e tratamento de
exceções em JAVA. No Capitulo \ref{cha:conclusao} é apresentada as
vantagens e desvantagens do suporte pela linguagem a concorrência e
exceções. 

\charpter{Concorrência}

Em alguns sistemas computacionais, processos que trabalham juntos podem compartilhar algum 
serviço em comum. Com efeito, a gestão da concorrência entre processos é a fonte de inúmeras 
dificuldades no desenvolvimento de software; o acesso descoordenado a um recurso (a chamada 
condição de corrida) induz no sistema um comportamento imprevisível.

Para evitar esse problema de recursos compartilhados devemos encontrar algum modo de impedir 
que mais de um processo utilize ao mesmo tempo estes recursos. Em outras palavras, precisamos 
de exclusão mútua, isto é, algum modo de assegurar que outros processos seja impedidos de usar 
um recurso que já estiver em uso por um processo.

Para realizar exclusão mútua de modo que, enquanto um processo estiver ocupado utilizando um 
recurso, nenhum outro processo cause problemas. Temos as seguintes alternativas:

\section{Lock}

Lock é uma solução de software para tentar impedir que dois processos acessem um recurso 
simultaneamente. Considere que haja uma única variável compartilhada (lock), inicialmente 
contendo o valor 0. Para entrar em sua região crítica, um processo testa antes se há 
impedimento, verificando o valor da variável lock. Se lock for 0, o processo altera essa 
variável para 1 e entra na região crítica. Se lock já estiver com o valor 1, o processo 
simplesmente aguardará até que ela se torne 0. Assim, um 0 significa que nenhum processo 
está utilizando o recurso e um 1 indica que algum processo está utilizando o recurso. 


\section{Lock reentrante}
….

\section{Spin}

Spin locks são um tipo especial de lock projetados para trabalhar em um ambiente 
multiprocessador. Se um processo encontra o spin lock aberto, ele adquiri o lock e entra. 
Caso contrário ele fica dando voltas, mesmo que ele não tenha nada pra fazer. Isso é bom 
pois a maioria dos recursos ficam trancadas por milisegundos apenas. Seria pior soltar a 
CPU e readiquiri-la mais tarde.

\section{Block}
….


\section{Semáforos}

Em 1965 Dijkstra sugeriu usar uma variável inteira para contar o número de vezes que o 
processo foi acordado. De acordo com a proposta dele, foi introduzido um novo tipo de 
variável chamado semáforo. Um semáforo poderia conter o valor 0 – indicando que nenhum 
sinal de acordar foi salvo – ou algum valor positivo se um ou mais sinais de acordar 
estivessem pendentes.

Dijkstra propôs a existência de duas operações,  douw e up. A operação down sobre um 
semáforo verifica se seu valor é maior que 0. Se for, o decrescerá de um e prosseguirá. 
Se o valor for 0, o processo será posto para dormir, sem terminar o down. Garante-se que, 
uma vez iniciada uma operação de semáforo, nenhum outro processo pode ter acesso ao 
semáforo até que a operação tenha terminado ou sido bloqueada. Essa atomicidade é 
absolutamente essencial para resolver os problemas de sincronização e evitar condições 
de disputa.

A operação up incrementa o valor de um dado semáforo. Se um ou mais processos estivessem 
dormindo naquele semáforo, incapacitados de terminar uma operação down anterior, um deles 
seria escolhido pelo sistema (por exemplo, aleatoriamente) e seria dada a permissão para 
terminar seu down. Portanto, depois de um up em um semáforo com processos dormindo nele, 
o semáforo permanecerá 0, mas haverá um processo a menor dormindo nele.

\section{Monitor}

Para facilitar a escrita correta de programas, Hoare e Brinch Hansen propuseram uma unidade 
básica de sincronização de alto nível chamada monitor.  Um monitor é uma coleção de procedimentos, 
variáveis e estruturas de dados, tudo isso agrupado em um tipo especial de módulo. Os processos 
podem chamar os procedimentos de um monitor quando quiserem, mas não podem ter acesso direto às 
estruturas internas de dados ao monitor a partir de procedimentos declarados fora do monitor.

Os monitores apresentam uma propriedade importante que os torna úteis para realizar a exclusão mútua: 
somente um processo pode estar ativo em um monitor em um dado momento. O monitor é uma construção da 
linguagem de programação. Em geral, quando um processo chama um procedimento do monitor, algumas das 
primeira instruções do procedimento verificarão  se qualquer outro processo está atualmente ativo 
dentro do monitor. Se estiver, o processo que chamou será suspenso até que o outro processo deixe o 
monitor. Se nenhum outro processo estiver usando o monitor, o processo que chamou poderá entrar.

\section{Deadlock}

Um conjunto de processos estará em situação de deadlock se todo processo pertencente ao conjunto 
estiver esperando por um evento que somente um outro processo desse mesmo conjunto poderá fazer 
acontecer. Como todos os processo estarão esperando, nenhum deles desencadeará qualquer um dos 
eventos que o outro esta esperando e, assim, todos os processos continuam a esperar para sempre.

Para que ocorra um Deadlock, deve ter quatro condições satisfeitas:

\begin{itemize}
\item Condição de exclusão mútua. Em um determinado instante, cada recurso está em uma de duas situações:
ou associado a um único processo ou disponível.
\item Condição de posse e espera. Processos que, em um determinado instante, retêm recursos concedidos
anteriormente podem requisitar novos recursos.
\item Condição de não preempção. Recursos concedidos previamente a um processo não podem ser 
forçosamente tomados desse processo - eles deve ser explicitamente liberados pelo processo que os retém.
\item Condição de espera circular. Deve existir um encadeamento circular de dois ou mais processos; cada um
deles encontra-se à espera de um recurso que está sendo usado pelo membro seguinte dessa cadeia.
\end{itemize}

Todas essas quatro condições deve estar presentes para que um deadlock ocorra. Se faltar
uma delas, não ocorrerá deadlock.

Em geral, quatro estratégias são usadas para tratar deadlocks:

\begin{itemize}
\item Ignorar por completo o problema.
\item Detecção e recuperação. Deixar os deadlocks ocorrer, detectá-los e agir.
\item Anulação dinâmica por meio de uma alocação cuidadosa de recursos.
\item Prevenção, negando estruturalmente algumas das condições necessárias para gerar um deadlock.
\end{itemize}

\section{Livelock}

Um Livelock é semelhante a um Deadlock, com exceção dos estados dos processos envolvidos na 
Livelock mudam constantemente em relação um ao outro, nenhum progresso Livelock é um caso 
especial de esgotamento de recursos.

Um exemplo do mundo real de Livelock ocorre quando duas pessoas se encontram em um corredor 
estreito, e cada um tenta ser educada, movendo de lado para  dar passagem a outra pessoa, 
mas eles acabam movendo de um lado para o outro sem fazer nenhum progresso porque ambos 
repetidamente movem na mesma direção e ao mesmo tempo.

\section{Inversão de Prioridade}

Considere um computador com dois processos: H, com alta prioridade, e L, com baixa prioridade. 
As regras de escalonamento são tais que H é executado sempre que estiver no estado pronto. 
Em certo momento, com L em sua região crítica, H torna-se pronto para executar. Agora H inicia 
uma espera ociosa, mas, como L nunca é escalonado enquanto H está executando, L nunca tem a 
oportunidade de deixar sua região crítica e, assim, H fica em um laço infinito. Essa situação 
é referida como problema da inversão de prioridade.

\section{Memória Transacional}

Memória Transacional atraiu
 grande interesse por eliminar muitos dos problemas associados ao uso de locks
 e ainda assim conseguir um ganho em desempenho.


Como exemplo, considere diversas tarefas que operam simultaneamente em uma mesma regiões 
de memória, realizando escritas e leituras. Uma modificação no estado feita por uma tarefa 
que e lida por outra pode denotar um estado inconsistente, se a modificação foi realizada 
durante a operação da segunda tarefa.

Uma transação seria uma sequencia de operações de leitura e escrita em memória compartilhada 
realizada por uma tarefa de forma otimista, isto e, assumindo que os estados intermediários 
não foram alterados
 concorrentemente por outras tarefas.


As operações realizadas nas transações entretanto são registradas, e logo antes do fim das 
transações e verificado se o estado lido está consistente, baseado nos registros; supondo 
que estejam, as modificações realizadas na transações o efetivadas, tornando-se permanentes. 
Caso contrário, o registro permite que quaisquer mudanças sejam desfeitas, abortando a transação.


\charpter{Exceções}

Uma exceção é um evento, normalmente associado a um erro, e que ao ser disparado interrompe o 
fluxo normal de execução da aplicação. Exceções ocorrem durante a execução de aplicativos, como 
por exemplo divisões por zero, acesso a posições inválidas (em listas ou arrays).


Erros são comuns em tempo de execução e as aplicações devem estar preparadas para tratá-los 
sem abortar o programa e perder os dados dos usuários. Erros são comuns em tempo de execução, 
as aplicações devem estar preparadas para tratá-los sem abortar o programa e perder os dados 
dos usuários. O conceito de tratamento de exceções permite ao programador ``lidar'' com o 
problema (tratando-o) e permitindo ao programa continuar sua execução.

As vantagens no uso de exceções são: 

\begin{itemize}
\item Separação do código regular (fluxo normal) do código de tratamento de exceções
\item Agrupamento e diferenciação de erros (e seus respectivos tratamentos)
\item Obrigatoriedade de tratamento (exceções não podem ser ignoradas)
\end{itemize}



\chapter{Bitcity}

XXX Aqui tem um parágrafo falando que usou Java porque é uma linguagem
bem legal que tem um modelo de concorrência e tratamento de exceções
joia e que para explorar tais recursos foi feita uma aplicação,
chamada de \textbf{Bitcity}.XXX

A \textbf{Bitcity} parte de um modelo textual para projetar um mundo
onde retângulos azuis representam carros que respeitam os semáforos
mas que não apreciam congestionamento, esporadicamente
retângulos brancos surgem e representam uma ambulância que age como um
coletor de recursos, folhas crescem em árvores e a chuva faz as mesmas
caírem.

O Modelo \ref{short-one} apresenta todos os elementos que podem ser
encontrados no modelo textual.

\begin{model}
\begin{verbatim}
9 18 3 2 15
######           #
*>  -#           #
$..$-#           # 
$..$A#############
*> A-     B>>
###$ ######B #####
  #$ #    #  #    
  #$ #    #+ #####
###$ #    #+    <*
\end{verbatim}
  \caption{Exemplo demonstrativo \label{short-one}}
\end{model}

A primeira linha contém quatro inteiros que descrevem, respectivamente,
quantidade de linhas, quantidade de colunas, número de pontos de
partida, número de conjuntos de semáforos a serem sincronizados e a
quantidade máxima de carros no mundo num dado instante.

O caractere \verb!`#'! representa calçada, \verb!`$'! representa
vegetação rasteira, \verb!`.'! uma árvore que ganha e perde folhas ao
longo da execução, uma letra qualquer que se repete indica semáforos
que pertencem a um mesmo conjunto. Os pontos de partida, que ficam
responsáveis por indicar os locais onde carros surgem no mundo, são
representados por \verb!`*'!. Os símbolos \verb!`+'!, \verb!`-'!,
\verb!`>'! e \verb!`<'! indicam, respectivamente, para cima, para
baixo, esquerda e direita e são utilizados pelos veículos para
movimentação no mundo.

O modelo é projetado de forma a simplificar a construção do mundo e
também o \textit{parsing} do mesmo. Todo ponto de partida precisa,
necessariamente, que uma direção inicial esteja presente ou
imediatamente a sua direita, ou a esquerda, ou em cima ou
embaixo. Essa direção define a direção de deslocamento inicial de
todos os carros que partirem daquele ponto. Os carros possuem uma
unidade de visão à sua frente, considerando sua direção atual. Esta
visão é utilizada para tomada de seis decisões:
\begin{description}
\item[Mudar direção atual]: O elemento à frente contém um dos
  caracteres que indicam uma direção e à frente deste outro caractere,
  considerando a direção que representa, o símbolo se repete. Esse
  caso ocorre no Modelo \ref{short-one} na linha 2, coluna 5.
\item[Escolher entre mudar de direção ou não]: O elemento à frente
  contém um dos caracteres que indicam uma direção e à frente deste
  outro caractere, considerando a direção que representa, o símbolo
  não se repete. Situação da linha 5, coluna 5 no Modelo \ref{short-one}.
\item[Parar no semáforo]: Veículo, exceto ambulância, depara-se com
  sinal vermelho e obrigatoriamente para.
\item[Buzinar]: O sinal está aberto, porém existe um veículo a sua
  frente que está parado.
\item[Sair do mundo]: O veículo atingiu o limite do mundo. No Modelo
  \ref{short-one} essa situação pode ocorrer somente ao atingir uma
  linha após a 9 na coluna 5 e também uma coluna após a 18 na linha 5.
\item[Deslocar-se]: Posição à frente não contém nenhum elemento
  especial, o veículo pode continuar deslocando-se seguindo sua
  direção corrente.
\end{description}

A aplicação espera iniciar sua execução com um modelo que se adeque as
especificações a cima. Caso isso não ocorra, a classe \verb!Parser!
fica responsável por detectar o problema e levantar uma exceção. A
classe \verb!Application! fica responsável por tratar esta exceção,
sendo exibido o erro encontrado e abortando a execução da aplicação.
Recebendo um modelo correto, a aplicação trata de renderizar o mundo
especificado. ...

\chapter{Conclusão}
\label{chap:conclusao}

1 página (+/- o limite)


\bibliographystyle{anbt-alfheng}
\bibliography{biblio}

\pagebreak

\end{document}
